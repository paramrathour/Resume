\documentclass[10pt, a4paper]{article}

\usepackage{microtype}
\usepackage{mathtools,bm}
\usepackage[utf8]{inputenc}                       % if you are not using xelatex ou lualatex, replace by the encoding you are using
\renewcommand{\familydefault}{\sfdefault}         % to set the default font; use '\sfdefault' for the default sans serif font, '\rmdefault' for the default roman one, or any tex font name
\usepackage{float}
\renewcommand*{\thefootnote}{\fnsymbol{footnote}}
\usepackage{array}
\newcolumntype{P}[1]{>{\centering\arraybackslash}p{#1}}
\usepackage{geometry}
\geometry{
	margin = 14.11mm, bmargin = 6.4mm, tmargin = 6.4mm
	%margin = 14.11mm, bmargin = 0.1in, tmargin = 0.4in
}
\usepackage{fancyhdr}
\fancypagestyle{firstpage}
{
	\headheight = 50.35mm
	\fancyhf{}
}
\renewcommand{\headrulewidth}{0pt}
\usepackage{graphicx}
\usepackage[dvipsnames]{xcolor}
\definecolor{DarkBlue}{RGB}{2,7,128}
\usepackage{titlesec}
\titleformat{\section}{\vspace{-1.4em}\bfseries\color{DarkBlue}\large}{}{0em}{}[\color{DarkBlue}\titlerule\vspace{-0.4em}]
\usepackage{import}
\usepackage{enumitem}
\usepackage{textcomp}
\setlist[itemize]{leftmargin=*}
\setlist[description]{leftmargin=*}
\usepackage[parfill]{parskip}
\setlength{\parindent}{0em}
\addtolength{\parskip}{-0.65em}
\renewcommand\labelitemi{$\vcenter{\hbox{\scriptsize$\bullet$}}$}
\newcommand{\lhs}[1]{{\textit{#1}}}
\newcommand{\lhsmall}[1]{{\small{\textit{#1}}}}
\newcommand{\rhs}[1]{\hfill{\small{\textsl{(#1)}}}}
\newcommand{\rhsmall}[1]{\hfill{\footnotesize{\textsl{(#1)}}}}
\newcommand{\rhse}[1]{\hfill{\small{\textsl{(#1)}}}\\[-12pt]}
\newcommand{\rhsmalle}[1]{\hfill{\footnotesize{\textsl{(#1)}}}\\[-12pt]}
\newcommand{\CFont}{\fontsize{11}{13.2}\selectfont}
\newcommand{\head}[1]{\vspace{0.2em}{\CFont{\textbf{#1}}}}

% personal data
\begin{document}
\thispagestyle{firstpage}
\pagenumbering{gobble}
Pursuing a \textbf{Minor} in \textbf{Computer Science \& Engineering}
\section{Scholastic Achievements}
\begin{itemize}
	\item Achieved a perfect \textbf{10 SPI} during the 8\textsuperscript{th} and 9\textsuperscript{th} semesters at IIT Bombay with 36 and 48 credits, respectively\rhsmall{2023}%(Semester Performance Index)
	\item Secured \textbf{All India Rank 926} in Joint Entrance Examination (\textbf{JEE}) \textbf{Advanced} among 161 thousand candidates\rhsmall{2019}
	\item Secured \textbf{99.9\%} percentile in Joint Entrance Examination (\textbf{JEE}) \textbf{Main} among 1.1 million candidates\rhsmall{2019}
	\item Recipient of the National Talent Search (\textbf{NTS}) Scholarship received by the top 1000 students in the country\rhsmall{2017}
	% \item Awarded Academic Excellence Scholarship (\textbf{AES}) by SOF given to a \textbf{single  student} per class in each state\rhsmall{2017}%Science Olympiad Foundation
	% \item Recipient of the Maharashtra Talent Search (\textbf{MTS}) Scholarship  with \textbf{State Rank 11, 10,} and \textbf{16} respectively\rhsmall{2015-17}%given by Centre for Talent Search and Excellence \\N. Wadia College, Pune
\end{itemize}
% \section{Work Experience}
% \vspace{-0.2em}
\section{Key Projects}
\head{NVIDIA $\mid$ Modeling the NVLink pipe ID in the GPU performance simulator}\rhsmalle{May 2022 - Jul 2022}\\
\lhsmall{Guide: Raghuram L}\rhsmalle{GPU Subsystem $\mid$ ASIC Intern}
% {\small\textbf{ASIC Intern} $\mid$ Modeling the \textbf{NVLINK} pipe ID in the GPU performance simulator using \textbf{PerfSim}}
\begin{itemize}
	% \item Explored modelling of the \textbf{NVLink} pipe ID using \textbf{PerfSim}, the GPU performance simulator developed using C\texttt{++} %building blocks, knobs, debugging, and architectural \& performance testing of models in C\texttt{++}
	\item Worked on enhancing the NVLink interconnect performance model to incorporate multiple pipes per High-Speed Hub%GPU-to-GPU
	\item Integrated a 1-D arbiter class template to the NVLink performance model while thoroughly maintaining its functionality
\end{itemize}
% \vspace{0.5em}
% \section{Key Projects}
% \vspace{-0.2em}
\head{Visual Learning and Recognition of 3-D Objects from Appearance}\rhsmall{Oct 2023 - Nov 2023}\\
\lhsmall{Guide: Prof. Ajit Rajwade}\rhsmalle{CS663 $\mid$ Fundamentals of Digital Image Processing $\mid$ Course Project}
\begin{itemize}
	\item Implemented a high-performance training and testing pipeline for object detection and pose estimation using \textbf{Python}% using the COIL-100 dataset
	% \item Utilised Principal Component Analysis \textbf{(PCA)} and cubic interpolation to construct parametric manifolds for each object
	% \item Conducted a comprehensive study across objects with varying complexities to determine \textbf{optimal} hyperparameter values
	\item Achieved an object recognition accuracy of \textbf{99.172\%} and a mean pose error of \textbf{6.872}$^{\bm\circ}$ by using the \textbf{COIL-100} dataset
\end{itemize}
\head{Efficient Cache Replacement Policy using Reinforcement Learning}\rhsmall{Sep 2023 - Nov 2023}\\
\lhsmall{Guide: Prof. Biswabandan Panda}\rhsmalle{CS683 $\mid$ Advanced Computer Architecture $\mid$ Course Project}
\begin{itemize}
	% \item Implemented Reinforcement Learned Replacement (RLR), an eviction policy based on {age, hit} and {type} priority of cache-lines%\textbf{cost-effective} 
	\item Designed Micro-Armed Bandit-based (MAB) replacement, utilising \textbf{temporal homogeneity} in the action space of policies%LRU, SHiP, SRRIP, DRRIP
	\item Evaluated both policies in \textbf{ChampSim} using 49 memory intensive traces from \textbf{SPEC} 2017 benchmarks and achieved an overall IPC speedup over LRU of \textbf{5\% }for RLR and \textbf{1.2\%} for MAB with LRU, SHiP, SRRIP, DRRIP in its action space
\end{itemize}
\head{Intelligent and Learning Agents}\rhsmall{Jul 2021 - Nov 2021}\\
\lhsmall{Guide: Prof. Shivaram Kalyanakrishnan}\rhsmalle{CS747 $\mid$ Foundations of Intelligent and Learning Agents $\mid$ Course Project}
\begin{itemize}
	% \item Implemented and compared $\mathsf{\varepsilon}$-greedy, \textbf{UCB}, KL-UCB and Thompson Sampling for a stochastic multi-armed bandit framework
	\item Performed \textbf{MDP Planning} using Value Iteration, Howard's Policy Iteration and Linear Programming with \textbf{PuLP} in Python
	\item Propelled up a car placed at the bottom of a sinusoidal valley using \textbf{Sarsa} with \textbf{Tile Coding} in the \textbf{OpenAI} Gym environment
\end{itemize}
\head{Autonomous Robotic Systems and Control}\rhsmall{Jan 2023 - May 2023}\\
\lhsmall{Guide: Prof. Debasattam Pal}\rhsmalle{EE615 $\mid$ Control and Computing Lab $\mid$ Course Project}
\begin{itemize}
	\item Realised \textbf{path planning} and \textbf{obstacle avoidance} of autonomous mobile robots in \textbf{MATLAB} using Vector Field Histogram
	\item Executed \textbf{sensor fusion} using complementary \& \textbf{Kalman filter} for estimating the orientation of inertial measurement units
	\item Implemented stabilisation of Rotary Inverted Pendulum using Swing-Up Control and \textbf{Linear-Quadratic Regulator} Control%Furuta pendulum%stabilization
\end{itemize}
\head{Coded Computing for Straggler Mitigation, Security and Privacy}\rhsmall{Sep 2021 - Nov 2021}\\
\lhsmall{Guide: Prof. Nikhil Karamchandani}\rhsmalle{EE605 $\mid$ Error Correcting Codes $\mid$ Course Project}
\begin{itemize}
	\item Investigated polynomial coding and Lagrange Coded Computing (LCC) techniques to mitigate fundamental bottlenecks in {Large-Scale Distributed Computing} for computing matrix multiplications and evaluating arbitrary multivariate polynomials
    \item Explored applications of LCC in secure \& private \textbf{Multi-Party Computing} (MPC) and \textbf{privacy-preserving} {machine learning}% (MPC) %Resilient
\end{itemize}
\head{Distributed Deep Learning}\rhsmall{Mar 2020 - Jul 2020}\\
\lhsmall{Institute Technical Summer Project (ITSP)}\rhsmalle{Institute Technical Council, IIT Bombay}
\begin{itemize}
	% \item Developed a \textbf{Hierarchically-Distributed} Deep CNN in order to parallelise workload across nodes in the learning model
	\item Developed a {Hierarchically-Distributed Deep CNN} learning model for training \textbf{super-high-resolution datasets} via {spatial segmentation} of each sample and observed an increase in \textbf{training speed} and a decrease in \textbf{memory utilisation} per node% in the hierarchy network
	% \item Compared the performance of \textbf{VGG16,  ResNet}, and \textbf{AlexNet} when used as the underlying Neural Networks%LeNet,DenseNet%state-of-the-art nets such as 	
	% \item Verified the approach by using Kaggle's \textbf{Retinal OCT} dataset and analysed loss of information due to spatial segmentation%\textbf{CINIC-10} Datasets on Kaggle %with accuracy more than 70\%
\end{itemize}
\section{Positions of Responsibility}
\vspace{-0.2em}
\head{IIT Bombay Racing $\mid$ Junior Design Engineer $\mid$ Electrical Subsystem}\rhsmalle{Sep 2020 - May 2021}
% {\small\textsl{Faculty Advisor: Prof. Amber Shrivastava}}\\% $\mid$ Dept. Of Mechanical Engineering, IIT Bombay}
% {\small\textsl{A cross-functional team of 70+ students which designs, fabricates and assembles an Electric Race Car for Formula Student (FS) UK}}\\[-12pt]% an international studnets race car} 
% \item \small\textsl{First Indian team to win the Engineering Design event in the history of FSUK (4\textsuperscript{th} overall out of 73 international teams)}
% {\small\textbf{Junior Design Engineer} $\mid$ LV Safety Subsystem}
\begin{itemize}
	% \item The subsytem controlled most of the Shutdown circuitry and detection of electrical faults in a car like \textbf{IMD} and \textbf{ECU} errors
	%\item Working on reducing board sizes of ciruits and improving testability and reliability of circuits
	\item Simulated the LV Safety board on \textbf{LTSpice} and verified the working of RTDS, brake light, and error blocks of the subsystem
	\item Explored Electromagnetic Interference (\textbf{EMI}) reduction techniques to be incorporated into PCB designs of the subsystem
	% \item \textbf{Mentored} 3 trainees in understanding the subsystem through the FS rulebook, circuit design tasks, and LTspice simulations
	% \item Doing integrated testing of the electrical components in the car
\end{itemize}
% \head{Teaching Assistant $\mid$ Computer Programming and Utilization $\mid$ Multivariable Control}\\
\head{Teaching Assistant $\mid$ Computer Programming and Utilisation}\rhsmalle{Autumn 2020, Autumn 2021, Spring 2022, Autumn 2022}
\begin{itemize}
	\item Academically guided \textbf{50} students, personally cleared their doubts, prepared and evaluated examinations \& lab problems
	\item Brainstormed \textbf{60+ practice problems} for CS101, shared via a personal \textbf{webpage} with tips and resources to boost interest% and more resources% to enhance interest%understanding
\end{itemize}
\section{Technical Skills}
\setlength\tabcolsep{0.3em}
\vspace{-0.3em}
\hspace{-5pt}
\begin{tabular}{p{1.15in}p{5.85in}}
\textbf{Languages}& C, C\texttt{++}, Python, Julia, MATLAB, Scilab, \LaTeX, HTML, CSS,  SQL, Embedded C, VHDL, MIPS, 8086\\%Assembly
\textbf{Frameworks}& Git, Docker, SageMath, Qiskit, NumPy, SciPy, pandas, scikit-learn, OpenCV, TensorFlow, Keras, Jekyll\\%Matplotlib% Selenium, Beautiful Soup, PyAutoGUI,%SymPy,%seaborn,%PyTesseract
% \textbf{Hardware}& Embedded C, VHDL, MIPS, 8051, 8086 Assembly, CPLD, Arduino, ESP32, Raspberry Pi 4, Tiva-C%, Pt-51%, Krypton\\
\end{tabular}
\section{Extracurriculars}
\vspace{-0.5em}
\begin{tabular}{p{0.99in}p{6.01in}}
\vspace{-0.5em}
\small\textbf{Volunteering} & \vspace{-0.5em}%{\tiny\textsl{(2019-2020)}}
\begin{itemize}
	% \item Conducted an institute-wide \textbf{Computer Programming} session (TSC) attended by {100+ students}\ \rhsmall{2022}% and discussing doubts \& previous year papers, organized by the Student Support Services, IIT Bombay%solving%
	\item Contributed to Career Counselling Campaign for 12,000+ indigent students by \textbf{Abhyuday}\ \rhsmall{2019}
	% \item \textbf{Mentored} JEE students during the \textbf{COVID-19} crisis as a part of \textbf{CovEd Education}\ \rhsmall{2020}% initiative%{\footnotesize\hfill\textsl{(2020)}}
\end{itemize}\\[-1em]\hline
\vspace{-0.5em}
\small\textbf{Miscellaneous}	& \vspace{-0.5em}
\begin{itemize}
	\item Composed articles on exciting labs and scientific content as an \textbf{Editor} of Department Newsletter\ \rhsmall{2020}%11 June 2020
	\item Completed a year-long \textbf{training program} as \textbf{NCC Cadet} under 2 MER NCC at IIT Bombay\ \rhsmall{2019}%{\footnotesize\hfill\textsl{(2019--20)}}
	% \item Part of the \textbf{Inter-Department E-Sports} Fest winning squad representing the \textbf{Smashkarts} team\ \rhsmall{2022}
\end{itemize}%\\[-1em]\hline
\end{tabular}
\end{document}
%% end of file `template.tex'.