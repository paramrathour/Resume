\documentclass[10pt, a4paper]{article}

\usepackage{microtype}
\usepackage{mathtools,bm}
\usepackage[utf8]{inputenc}                       % if you are not using xelatex ou lualatex, replace by the encoding you are using
\renewcommand{\familydefault}{\sfdefault}         % to set the default font; use '\sfdefault' for the default sans serif font, '\rmdefault' for the default roman one, or any tex font name
\usepackage{float}
\renewcommand*{\thefootnote}{\fnsymbol{footnote}}
\usepackage{array}
\newcolumntype{P}[1]{>{\centering\arraybackslash}p{#1}}
\usepackage{geometry}
\geometry{
	margin = 14.11mm, bmargin = 1.4mm, tmargin = 6.4mm
	%margin = 14.11mm, bmargin = 0.1in, tmargin = 0.4in
}
\usepackage{fancyhdr}
\fancypagestyle{firstpage}
{
	\headheight = 60mm
	\fancyhf{}
}
\renewcommand{\headrulewidth}{0pt}
\usepackage{graphicx}
\usepackage[dvipsnames]{xcolor}
\definecolor{DarkBlue}{RGB}{2,7,128}
\usepackage{titlesec}
\titleformat{\section}{\vspace{-1.4em}\bfseries\color{DarkBlue}\Large}{}{0em}{}[\color{DarkBlue}\titlerule\vspace{-0.2em}]
\usepackage{import}
\usepackage{enumitem}
\usepackage{textcomp}
\setlist[itemize]{leftmargin=*}
\setlist[description]{leftmargin=*}
\usepackage[parfill]{parskip}
\setlength{\parindent}{0em}
\addtolength{\parskip}{-0.55em}
\renewcommand\labelitemi{$\vcenter{\hbox{\scriptsize$\bullet$}}$}
\newcommand{\lhs}[1]{{\textit{#1}}}
\newcommand{\lhsmall}[1]{{\small{\textit{#1}}}}
\newcommand{\rhs}[1]{\hfill{\small{\textsl{(#1)}}}}
\newcommand{\rhsmall}[1]{\hfill{\footnotesize{\textsl{(#1)}}}}
\newcommand{\rhse}[1]{\hfill{\small{\textsl{(#1)}}}\\[-12pt]}
\newcommand{\rhsmalle}[1]{\hfill{\footnotesize{\textsl{(#1)}}}\\[-12pt]}
\newcommand{\CFont}{\fontsize{11}{13.2}\selectfont}
\newcommand{\head}[1]{\vspace{0.2em}{\CFont{\textbf{#1}}}}

% personal data
\begin{document}
\thispagestyle{firstpage}
\pagenumbering{gobble}
Completed a \textbf{Minor} in \textbf{Computer Science \& Engineering}
\section{Scholastic Achievements}
\begin{itemize}
	\item Awarded the \textbf{Academic Excellence Award} for \textbf{ranking first} in the Control and Computing specialization at IIT Bombay%\rhsmall{2024}
	\item Achieved a perfect \textbf{10 SPI} during the 8\textsuperscript{th} and 9\textsuperscript{th} semesters at IIT Bombay with 36 and 48 credits, respectively\rhsmall{2023}%(Semester Performance Index)
	\item Awarded an \textbf{AP} grade for exceptional performance in the Advanced Computer Architecture course at IIT Bombay\rhsmall{2023}
	\item Secured \textbf{All India Rank 926} in Joint Entrance Examination (\textbf{JEE}) \textbf{Advanced} among 161 thousand candidates\rhsmall{2019}
	\item Secured \textbf{99.9\%} percentile in Joint Entrance Examination (\textbf{JEE}) \textbf{Main} among 1.1 million candidates\rhsmall{2019}
	\item Recipient of the National Talent Search (\textbf{NTS}) Scholarship received by the top 1000 students in the country\rhsmall{2017}
	\item Awarded Academic Excellence Scholarship (\textbf{AES}) by SOF given to a \textbf{single  student} per class in each state\rhsmall{2017}%Science Olympiad Foundation
	% \item Recipient of the Maharashtra Talent Search (\textbf{MTS}) Scholarship  with \textbf{State Rank 11, 10,} and \textbf{16} respectively\rhsmall{2015-17}%given by Centre for Talent Search and Excellence \\N. Wadia College, Pune
\end{itemize}
\section{Work Experience}
\vspace{-0.2em}
\head{NVIDIA $\mid$ GPU Subsystem}\\
\lhsmall{Guide: Raghuram L}\\
{\small\textbf{ASIC Intern} $\mid$ Modelling the \textbf{NVLink} pipe ID in the GPU performance simulator}\rhsmalle{May 2022 - Jul 2022}
\begin{itemize}
	\item Explored \textbf{PerfSim} building blocks, knobs, debugging, and architectural \& performance testing of models in C\texttt{++}
	\item Worked on enhancing the NVLink interconnect performance model to incorporate multiple pipes per High-Speed Hub%GPU-to-GPU
	\item Integrated a 1-D arbiter class template to the NVLink performance model while thoroughly maintaining its functionality
\end{itemize}
\head{IIT Bombay Racing $\mid$ Electrical Subsystem}\\
% {\small\textsl{Faculty Advisor: Prof. Amber Shrivastava}}\\% $\mid$ Dept. Of Mechanical Engineering, IIT Bombay}
{\small\textsl{A cross-functional team of 70+ students which designs, fabricates and assembles an Electric Race Car for Formula Student (FS) UK}}\\% an international studnets race car} 
% \item \small\textsl{First Indian team to win the Engineering Design event in the history of FSUK (4\textsuperscript{th} overall out of 73 international teams)}
{\small\textbf{Junior Design Engineer} $\mid$ LV Safety Subsystem}\rhsmalle{Sep 2020 - May 2021}
\begin{itemize}
	% \item The subsytem controlled most of the Shutdown circuitry and detection of electrical faults in a car like \textbf{IMD} and \textbf{ECU} errors
	%\item Working on reducing board sizes of ciruits and improving testability and reliability of circuits
	\item Simulated the LV Safety board on \textbf{LTspice} and verified the working of RTDS, brake light, and error blocks of the subsystem
	\item Explored Electromagnetic Interference (\textbf{EMI}) reduction techniques to be incorporated into PCB designs of the subsystem
	\item \textbf{Mentored} 3 trainees in understanding the subsystem through the FS rulebook, circuit design tasks, and LTspice simulations
	% \item Doing integrated testing of the electrical components in the car
\end{itemize}
{\small\textbf{Trainee} \small$\mid$ Electrical Subsystem}\rhsmalle{Jan 2020 - Aug 2020)}
\begin{itemize}
	% \item Explored the \textbf{LV Safety} subsystem, the \textbf{Shutdown Sequence} of the car and its elements
	\item Investigated the {Electronic Control Unit} (\textbf{ECU}) subsystem, working with {RPM} and {position sensors} and realised the working of the steering, acceleration pedal and brake sensors of the car with \textbf{Arduino IDE} (Integrated Development Environment)
	\item Acquired knowledge of {Controller Area Network} (\textbf{CAN}) protocol \& {Data Acquisition} (\textbf{DAQ}) system and their implementation, programmed code for wireless communication using \textbf{LPC1768 Mbed} microcontroller and \textbf{XBee} module
\end{itemize}
% \vspace{0.5em}
\section{Key Projects}
\vspace{-0.2em}
\head{Data-Driven Control using Informativity in Presence of Adversarial Attacks}\rhsmall{Jul 2023 - Jul 2024}\\ 
\lhsmall{Guide: Prof. Debasattam Pal}\rhsmalle{Dual Degree Project, IIT Bombay}
\begin{itemize}
	\item Explored the \textbf{Behavioral Approach} in control to develop suitable mathematical framework for analysing dynamical systems
	\item Investigated into inferring the dissipativity properties of linear systems from measured data using the \textbf{Informativity} framework
	\item Examined threat models for an \textbf{adversarial attacker} and worked on verifying system properties using corrupted data
\end{itemize}
\head{Computational Commutative Algebra and Geometry}\rhsmall{Jul 2022 - Nov 2022}\\ 
\lhsmall{Guide: Prof. Debasattam Pal}\rhsmalle{Supervised Research Exposition}
\begin{itemize}
	\item Investigated into the theory and computation of \textbf{Gr\"obner Bases} for Ideals in a polynomial ring ${k[x_1,\ldots,x_n]}$ over a field ${F}$
	\item Explored the algebraic and geometric applications of Gr\"obner Bases in solving Ideals, Varieties and Nullstellensatz problems
	\item Implemented fast solvers for system of linear \& polynomial equations and Sudoku in \textbf{SageMath} using Elimination Theory
\end{itemize}
\head{Autonomous Robotic Systems and Control}\rhsmall{Jan 2023 - May 2023}\\
\lhsmall{Guide: Prof. Debasattam Pal}\rhsmalle{EE615 $\mid$ Control and Computing Lab $\mid$ Course Project}
\begin{itemize}
	\item Realised \textbf{path planning} and \textbf{obstacle avoidance} of autonomous mobile robots in \textbf{MATLAB} using Vector Field Histogram
	\item Executed \textbf{sensor fusion} using complementary \& \textbf{Kalman filter} for estimating the orientation of inertial measurement units
	\item Implemented stabilisation of Rotary Inverted Pendulum using Swing-Up Control and \textbf{Linear-Quadratic Regulator} Control%Furuta pendulum%stabilization
\end{itemize}
\head{Intelligent and Learning Agents}\rhsmall{Jul 2021 - Nov 2021}\\
\lhsmall{Guide: Prof. Shivaram Kalyanakrishnan}\rhsmalle{CS747 $\mid$ Foundations of Intelligent and Learning Agents $\mid$ Course Project}
\begin{itemize}
	\item Implemented and compared $\mathsf{\varepsilon}$-greedy, \textbf{UCB}, KL-UCB and Thompson Sampling for a stochastic multi-armed bandit framework
	\item Performed \textbf{MDP Planning} using Value Iteration, Howard's Policy Iteration and Linear Programming with \textbf{PuLP} in Python
	\item Propelled up a car placed at the bottom of a sinusoidal valley using \textbf{Sarsa} with \textbf{Tile Coding} in the \textbf{OpenAI} Gym environment
\end{itemize}
% \head{Dining Philosophers: A Synchronisation Problem}\rhsmall{Jan 2022 - May 2022}\\
% \lhsmall{Guide: Prof. Mythili Vutukuru}\rhsmalle{CS347 $\mid$ Operating Systems $\mid$ Course Project}
% \begin{itemize}
% 	\item Modelled the threads by creating custom semaphores using condition variables and mutex abstractions of \textbf{pthreads} API
% 	\item Devised and implemented two solutions by using \textbf{semaphores} and \textbf{condition variables} each and proved their correctness
% \end{itemize}
\clearpage
\head{Visual Learning and Recognition of 3-D Objects from Appearance}\rhsmall{Oct 2023 - Nov 2023}\\
\lhsmall{Guide: Prof. Ajit Rajwade}\rhsmalle{CS663 $\mid$ Fundamentals of Digital Image Processing $\mid$ Course Project}
\begin{itemize}
	\item Implemented a high-performance training and testing pipeline for object detection and pose estimation using \textbf{Python}% using the COIL-100 dataset
	\item Utilised Principal Component Analysis \textbf{(PCA)} and cubic interpolation to construct parametric manifolds for each object
	% \item Conducted a comprehensive study across objects with varying complexities to determine \textbf{optimal} hyperparameter values
	\item Achieved an object recognition accuracy of \textbf{99.172\%} and a mean pose error of \textbf{6.872}$^{\bm\circ}$ by using the \textbf{COIL-100} dataset
\end{itemize}
\head{Efficient Cache Replacement Policy using Reinforcement Learning}\rhsmall{Sep 2023 - Nov 2023}\\
\lhsmall{Guide: Prof. Biswabandan Panda}\rhsmalle{CS683 $\mid$ Advanced Computer Architecture $\mid$ Course Project}
\begin{itemize}
	\item Implemented Reinforcement Learned Replacement (RLR), an eviction policy based on {age, hit} and {type} priority of cache-lines%\textbf{cost-effective} 
	\item Designed Micro-Armed Bandit-based (MAB) replacement, utilising \textbf{temporal homogeneity} in the action space of policies%LRU, SHiP, SRRIP, DRRIP
	\item Evaluated both policies in \textbf{ChampSim} using 49 memory intensive traces from \textbf{SPEC} 2017 benchmarks and achieved an overall IPC speedup over LRU of \textbf{5\% }for RLR and \textbf{1.2\%} for MAB with LRU, SHiP, SRRIP, DRRIP in its action space
\end{itemize}
\head{Coded Compressed Sensing Scheme for Unsourced Multiple Access}\rhsmall{Mar 2024 - May 2024}\\
\lhsmall{Guide: Prof. Ajit Rajwade}\rhsmalle{CS754 $\mid$ Advanced Image Processing}
\begin{itemize}
	\item Simulated an \textbf{uncoordinated} message transmission scheme to a single access point using \textbf{tree encoding} and \textbf{decoding}
	\item Utilised compressed sensing %measurement matrices
	to transform signals into %binary
	sparse vectors and decoded them with \textbf{Orthogonal Matching Pursuit}% \textbf{(OMP)}
	\item Balanced the optimisation objectives of the per-user error probability and computational complexity using \textbf{CVXPY} framework
	\item Achieved a partial signal recovery of synthetic messages by an accurate reconstruction of \textbf{30\%-40\%} of sent bits' prefixes
\end{itemize}
\head{Coded Computing for Straggler Mitigation, Security and Privacy}\rhsmall{Sep 2021 - Nov 2021}\\
\lhsmall{Guide: Prof. Nikhil Karamchandani}\rhsmalle{EE605 $\mid$ Error Correcting Codes $\mid$ Course Project}
\begin{itemize}
	\item Investigated polynomial coding and Lagrange Coded Computing (LCC) techniques to mitigate fundamental bottlenecks in \textbf{Large-Scale Distributed Computing} for computing matrix multiplications and evaluating arbitrary multivariate polynomials
    \item Explored applications of LCC in secure \& private \textbf{Multi-Party Computing} (MPC) and \textbf{privacy-preserving} {machine learning}% (MPC) %Resilient
\end{itemize}
\head{Pushdown Timed Automata: Theory and Practice}\rhsmall{May 2022 - Dec 2022}\\
\lhsmall{Guide: Prof. Akshay S.}\rhsmalle{CS490 $\mid$ Research and Development $\mid$ Academic Project}
\begin{itemize}
	\item Conceptualized modelling problems for Pushdown Timed Automata (PDTA) from {Embedded Systems} and {WCET Benchmarks}%Worst-case execution time
	\item Conducted intensive review of various tools for the simulation and \textbf{reachability analysis} of Pushdown Automata \& PDTA%Pushdown Automata
	\item Developed methodology to extract Pushdown Systems of \textbf{Boolean} and \textbf{Remopla} programs using \textbf{Moped} Model Checker
\end{itemize}
% \head{Data-Driven Dynamical Systems}\rhsmall{Jan 2023 - May 2023}\\
% \lhsmall{Guide: Prof. Vivek Borkar}\rhsmalle{EE736 $\mid$ Stochastic Optimization $\mid$ Course Project}
% \begin{itemize}
% 	\item Reviewed the paradigms of Koopman Theory, Dynamic Mode Decomposition (\textbf{DMD}) and Extended DMD with control
% 	\item Examined the ideas for discovering governing equations from data by Sparse Identification of Nonlinear Dynamics (\textbf{SINDy})
% 	\item Investigated into {Compressed Sensing} and \textbf{Sparse Regression} techniques for solving the intermediate stages of SINDy
% \end{itemize}
% \clearpage
% \section{Key Projects}
% \vspace{-0.2em}
% \clearpage
\head{Distributed Deep Learning}\rhsmall{Mar 2020 - Jul 2020}\\
\lhsmall{Institute Technical Summer Project (ITSP)}\rhsmalle{Institute Technical Council, IIT Bombay}
\begin{itemize}
	% \item Developed a \textbf{Hierarchically-Distributed} Deep CNN in order to parallelise workload across nodes in the learning model
	\item Developed a {Hierarchically-Distributed Deep CNN} learning model for training \textbf{super-high-resolution datasets} via {spatial segmentation} of each sample and observed an increase in \textbf{training speed} and a decrease in \textbf{memory utilisation} per node% in the hierarchy network
	% \item Compared the performance of \textbf{VGG16,  ResNet}, and \textbf{AlexNet} when used as the underlying Neural Networks%LeNet,DenseNet%state-of-the-art nets such as 	
	\item Verified the approach by using Kaggle's \textbf{Retinal OCT} dataset and analysed loss of information due to spatial segmentation%\textbf{CINIC-10} Datasets on Kaggle %with accuracy more than 70\%
\end{itemize}
% \head{Two-Way Fetch Superscalar Processor}\rhsmall{Jan 2022 - May 2022}\\
% \lhsmall{Guide: Prof. Virendra Singh}\rhsmalle{EE739 $\mid$ Processor Design $\mid$ Course Project}
% \begin{itemize}
% 	\item Designed a \textbf{six-stage} 16-bit superscalar processor capable of handling \textbf{19} arithmetic, logical, and branching instructions
% 	\item Employed two-way instruction fetch, decode, dispatch, execute and write-back stages with \textbf{branch prediction} techniques
% 	\item Designed a \textbf{16-bit} \textbf{signed ALU} implementing addition using \textbf{Kogge-Stone} fast adder and verified it using Intel Quartus %Prime%high performing
% \end{itemize}
% \head{Cryptanalysis of Pseudorandom Generators}\rhsmall{Jan 2023 - May 2023}\\
% \lhsmall{Guide: Prof. Virendra Sule}\rhsmalle{EE720, EE793 $\mid$ Cryptology $\mid$ Course Project}
% \begin{itemize}%Pseudorandom Bit Multisequences
% 	\item Analysed Linear Complexity (LC) profiles of the bit multi-sequences with \textbf{3-SAT}, Quadratic Residue and Exponential Map
% 	\item Implemented reduced-\textbf{Moustique}, a self-synchronising stream cipher, achieving \textbf{almost perfect LC} profiles in \textbf{SageMath}
% \end{itemize}
\section{Positions of Responsibility}
\vspace{-0.2em}
\head{Teaching Assistant $\mid$ IIT Bombay}\\
% \lhsmall{Guide: Prof. Bhaskaran Raman, Prof. Parag Chaudhuri, Prof. Akshay S., Prof. Ajit Rajwade, Prof Madhu Belur}\\
{\small\textbf{Computer Programming and Utilisation $\mid$ CS101}} \rhsmalle{Autumn 2020, Autumn 2021, Spring 2022, Autumn 2022}
\begin{itemize}
	\item Academically guided \textbf{50+} students and cleared their doubts through weekly doubt sessions, labs and personal interaction
	\item Prepared and evaluated examinations \& lab problems and conducted Hindi help sessions for students facing language barriers
	\item Brainstormed \textbf{60+ practice problems} for CS101, shared via a personal \textbf{webpage} with tips and resources to boost interest% and more resources% to enhance interest%understanding
\end{itemize}
{\small\textbf{Multivariable Control $\mid$ EE640}}\rhsmalle{Autumn 2023}
\begin{itemize}
	\item Academically guided \textbf{40+} students, clearing their doubts through tutorials and assisting the instructor in course logistics
\end{itemize}
\head{Mentor $\mid$ Summer of Science}\rhsmall{Summer 2021, Summer 2022, Summer 2023, Summer 2024}\\
\lhsmall{Topics: DSA, Linear Algebra, Cryptography, Coding Theory, Reinforcement Learning}\rhsmalle{Maths and Physics Club, IIT Bombay}
\begin{itemize}
	\item Mentored \textbf{ten students} in exploring the subject, cleared their doubts, reviewed and evaluated their reports \& presentations
\end{itemize}
\section{Technical Skills}
\setlength\tabcolsep{0.3em}
\vspace{-0.3em}
\hspace{-5pt}
\begin{tabular}{p{1.15in}p{5.85in}}
\textbf{Languages}& C, C\texttt{++}, Python, Julia, MATLAB, Scilab, \LaTeX, HTML, CSS,  SQL, Embedded C, VHDL, MIPS, 8086\\%Assembly
\textbf{Frameworks}& Git, Docker, SageMath, Qiskit, NumPy, SciPy, pandas, scikit-learn, OpenCV, TensorFlow, Keras, Jekyll\\%Matplotlib% Selenium, Beautiful Soup, PyAutoGUI,%SymPy,%seaborn,%PyTesseract
\textbf{Software}& Simulink, EAGLE, LTspice, Intel Quartus, Keil $\mu$Vision, GNURadio, Adobe Illustrator, SOLIDWORKS\\%%Simulink
% \textbf{Hardware}& Embedded C, VHDL, MIPS, 8051, 8086 Assembly, Arduino, ESP32, Raspberry Pi 4, Tiva-C%, Pt-51%, Krypton\\
\end{tabular}
\section{Key Courses Undertaken}
\setlength\tabcolsep{0.3em}
\vspace{-0.3em}
\hspace{-5pt}
\begin{tabular}{p{1.15in}p{5.85in}}
%{\linespread{1.8}
\textbf{Electrical} & Advanced Computer Architechture, {Digital Systems}, {Signal Processing}, Information Theory and Coding\\%, {Communication Systems}, Electronic Devices, Introduction to Electrical Engineering Practice \\%+Lab%Microprocessors%Dynamical%Systems
\textbf{Computer Science}& Operating Systems, Computer Networks, Data Structures, Design and Analysis of Algorithms,\newline Advanced Image Processing, Intelligent and Learning Agents, Formal Methods in Machine Learning\\%, Computer Programming and Utilization\\%{Introduction to Machine Learning}
\textbf{Mathematics}& Linear Algebra, %{Matrix Computations}, 
Large Sparse Matrix Computations, Game Theory and Algorithmic Mechanism Design,\newline Probability and Random Processes, Stochastic Optimisation, Logic, Number Theory and Cryptography\\%An Introduction to 
%\textbf{Physics}& Quantum Physics and Application, Basics of Electricity and Magnetism                 \\
% \textbf{Coursera}& {Deep Learning Specialization} (deeplearning.ai)\\% by deeplearning.ai 
\textbf{Bootcamps}& Data Analytics, Scientific Computation and Mathematical Modelling, Quantum Computing, Tinkering
\end{tabular}
% \begin{description}
% \item {\hfill\footnotesize\textsuperscript\textdagger\textsl{to be completed by May 2024}}
% \end{description}
% \vspace{-1.4em}
\section{Extracurriculars}
\vspace{-0.5em}
\begin{tabular}{p{0.99in}p{6.01in}}
\vspace{-0.5em}
% \vspace{-0.5em}
% \small\textbf{Technical} & %\newline{\scriptsize\textsl{(2019-2021)}} 	& 
% \vspace{-0.5em}
% \begin{itemize}
% 	% \item Built a \textbf{RC Bot} capable of negotiating obstacles and designed \& fabricated a \textbf{RC Trainer Plane}%\rhs{2019}
% 	% \item Completed \textbf{Scientific Computing} \& \textbf{Data Analytics} Bootcamps and \textbf{Quantum Computing} Workshop%\rhs{2020}
% 	\item Qualified Round 1 of \textbf{Mathathon} conducted by Maths and Physics (MnP) Club, IIT Bombay\ \rhsmall{2021}
% 	\item Completed Summer of Science in \textbf{Nonlinear Dynamics} and \textbf{Game Theory} by MnP Club\ \rhsmall{2020, 2021}
% \end{itemize}\\[-1em]\hline
% \small\textbf{NCC}\newline{\scriptsize\textsl{(2019-2020)}} 	& 
% \vspace{-0.5em}
% \begin{itemize}
% 	\item Completed a year-long \textbf{training program} as \textbf{NCC Cadet} under 2 MER NCC at IIT Bombay%{\footnotesize\hfill\textsl{(2019--20)}}
% 	\item Attended ten-day-long NCC \textbf{Annual Training Camp} (ATC) held during November-December 2019
% 	\item Part of \textbf{Republic Day Parade Contingent} held on 26\textsuperscript{th} January 2020 at IIT Bombay Gymkhana Ground
% 	\item Represented IIT Bombay in \textbf{Inter-College Cricket} Competition at (ATC) organised by NCC
% 	\item Participated in \textbf{Group Act Competition}, Cultural GC organised by NCC IIT Bombay	
% \end{itemize}\\[-1em]\hline
% \vspace{-0.5em}
\small\textbf{Volunteering} & \vspace{-0.5em}%{\tiny\textsl{(2019-2020)}}
\begin{itemize}
	\item Conducted an institute-wide \textbf{Computer Programming} session (TSC) attended by {100+ students}\ \rhsmall{2022}% and discussing doubts \& previous year papers, organized by the Student Support Services, IIT Bombay%solving%
	\item Contributed to Career Counselling Campaign for 12,000+ indigent students by \textbf{Abhyuday}\ \rhsmall{2019}
	\item \textbf{Mentored} JEE students during the \textbf{COVID-19} crisis as a part of \textbf{CovEd Education}\ \rhsmall{2020}% initiative%{\footnotesize\hfill\textsl{(2020)}}
\end{itemize}\\[-1em]\hline
\vspace{-0.5em}
\small\textbf{Miscellaneous}	& \vspace{-0.5em}
\begin{itemize}
	\item Composed articles on exciting labs and scientific content as an \textbf{Editor} of Department Newsletter\ \rhsmall{2020}%11 June 2020
	\item Completed a year-long \textbf{training program} as \textbf{NCC Cadet} under 2 MER NCC at IIT Bombay\ \rhsmall{2019}%{\footnotesize\hfill\textsl{(2019--20)}}
	\item Part of the \textbf{Inter-Department E-Sports} Fest winning squad representing the \textbf{Smashkarts} team\ \rhsmall{2022}
\end{itemize}%\\[-1em]\hline
% \vspace{-0.5em}
% \small\textbf{Sports}	& \vspace{-0.5em}
% \begin{itemize}
% 	\item Secured 2\textsuperscript{nd} runner-up in the \textbf{Inter-IIT Scrabble} League representing the IIT Bombay contigent\ \rhsmall{2020}%11 June 2020
% 	\item Part of the \textbf{Inter-Department E-Sports} Fest winning squad representing the \textbf{Smashkarts} team\ \rhsmall{2022}
% 	% \item Awarded the title of \textbf{Best Smashkarts Player} by Electrical Engineering Students Association\  \rhsmall{2022}
% 	\item Represented IIT Bombay in \textbf{Inter-College Cricket} Competition organised by NCC, IIT Bombay\ \rhsmall{2019}
% \end{itemize}%\\[-1em]\hline
% \vspace{-0.5em}
% \small\textbf{NCC}	& \vspace{-0.5em}
% \begin{itemize}
% 	\item Completed a year-long \textbf{training program} as \textbf{NCC Cadet} under 2 MER NCC at IIT Bombay\ \rhsmall{2019}%{\footnotesize\hfill\textsl{(2019--20)}}
% 	\item Attended a ten-day-long \textbf{Annual Training Camp} (ATC) organised by NCC, IIT Bombay\ \rhsmall{2019}
% 	\item Part of \textbf{Republic Day Parade} Contingent held on 26\textsuperscript{th} January 2020 at IIT Bombay Gymkhana\ \rhsmall{2020}
% \end{itemize}\\[-1em]\hline
% \vspace{-0.5em}
% \small\textbf{Culturals} & \vspace{-0.5em}
% \begin{itemize}
% 	\item Participated in \textbf{Group Act Competition}, Cultural GC organised by NCC IIT Bombay\ \rhsmall{2019}
% 	\item Studied \textbf{Beginner Music Theory} as a part of Summer School of Cult conducted by ICC\ \rhsmall{2020}
% \end{itemize}
\end{tabular}
\end{document}
%% end of file `template.tex'.